%
% Einfache LaTeX-Vorlage f�r Arbeiten am Lehrstuhl Kranzlm�ller / MNM-Team
% - optimiert f�r die Arbeit mit g�ngigen LaTeX-Editoren
% - funktioniert ohne Makefile und Anpassungen der LaTeX-Verzeichnisstruktur
% - verwendet Komaskript f�r ein (nach europ�ischen Gepflogenheiten) sch�neres Layout
% 
% v1, 2007 (Michael Brenner)
% Diese Version: v1.1, 2012 (Michael Brenner)
% Diese Version: v1.2, 2017 (Michael Brenner)
% 


\documentclass[bibliography=totoc,listof=totoc,BCOR=5mm,DIV=12]{scrbook} % Rand f�r Bindung: 5mm / falls Index verwendet, erg�nze "index=totoc" zu den Optionen 
\usepackage{ngerman} % Trennung nach neuer deutscher Rechtschreibung, deutsche Sonderzeichen, z.B. \glqq und \grqq f�r deutsche Anf�hrungszeichen 
\usepackage[latin1]{inputenc} % Umlaute im Text
\usepackage{graphicx} % Einf�gen von Grafiken  - f�r PDF-Latex: .pdf und .png (.jpg m�glich, sollte aber vermieden werden)
\usepackage{url}           % URL's (z.B. in Literatur) sch�ner formatieren
\usepackage{hyperref} % sorgt f�r f�r Hyperlinks in PDF-Dokumenten
\usepackage{color}

\graphicspath{{./Bilder/}}

\input{hyphenation} % in dieses File kommen W�rter die Latex nicht richtig trennt

\begin{document}

% ---------------------------------------------------------------
\frontmatter % Titelbl�tter und Erkl�rung jeweils spezifisch f�r die jeweilige Uni einbinden
    %%%%%%%%%%%%%%%%%%%%%%%%%%%%%%%
% erste Seite

\thispagestyle{empty}

\begin{center}

\vspace*{-2cm}

{\Huge INSTITUT F�R INFORMATIK\\[1mm]}
DER LUDWIG--MAXIMILIANS--UNIVERSIT�T M�NCHEN\\

\vspace*{1cm}

\includegraphics[width=0.3\textwidth]{lmu_siegel}

\vspace*{2cm}

{\Large \textbf{Masterarbeit}}\\ % oder Fortgeschrittenenpraktikum, Master's Thesis, Bachelorarbeit etc.

\vspace{2.0cm}
{\Huge \textbf{Dynamic PGAS Data}}\\
\vspace*{3mm}
{\Huge \textbf{Structures}}\\
\vspace*{20mm}

{\LARGE Stefan Effenberger} % Name des Autors

\vspace{3cm}
\end{center}

\newpage

%%%%%%%%%%%%%%%%%%%%%%%%%%%%%%%
% zweite Seite

\thispagestyle{empty}
\cleardoublepage

%%%%%%%%%%%%%%%%%%%%%%%%%%%%%%%
% dritte Seite (Kopie der ersten)

\thispagestyle{empty}

\begin{center}

\vspace*{-2cm}

{\Huge INSTITUT F�R INFORMATIK\\[1mm]}
DER LUDWIG--MAXIMILIANS--UNIVERSIT�T M�NCHEN\\

\vspace*{1cm}

\includegraphics[width=0.3\textwidth]{lmu_siegel}

\vspace*{2cm}

{\Large \textbf{Masterarbeit}}\\ % oder Fortgeschrittenenpraktikum, SEP etc.

\vspace{2.0cm}
{\Huge \textbf{Dynamic PGAS Data}}\\
\vspace*{3mm}
{\Huge \textbf{Structures}}\\
\vspace*{20mm}

{\LARGE Stefan Effenberger} % Name des Autors
\vspace{2cm}

\parbox{1cm}{
\begin{large}
\begin{tabbing}
Aufgabensteller: \hspace{.5cm} \=Prof. Dr. Dieter Kranzlm�ller\\[2mm]
Betreuer:
\>Tobias Fuchs\\[5mm]
Abgabetermin: \> \textcolor{red}{ADD DATE}\\
\end{tabbing}
\end{large}}\\
\vspace{5mm}

\end{center}
 % Titelbl�tter LMU - auskommentieren falls TUM-Arbeit
%    % Richtlinien, siehe http://wwwpa.in.tum.de/generell/Abschlussarbeitsform.html
%
%%%%%%%%%%%%%%%%%%%%%%%%%%%%%%%


% Deckblatt

\thispagestyle{empty}

\begin{center}
    \includegraphics[width=3cm]{tum-logo}\\
    \vspace{.5cm}
% "Technische Universit�t M�nchen" oder alternativ das Logo der TUM
    {\Large \sc Technische Universit�t M�nchen}\\

    \vspace{1cm}
% "Fakult�t f�r Informatik"
    {\Huge \sc Fakult�t f�r Informatik\\[1mm]}


    \vspace{2cm}
% Diplomarbeit | Master's Thesis | Bachelorarbeit in Informatik | Wirtschaftsinformatik |
    {\Large \textbf{Masterarbeit in Informatik}}\\
% Thema bzw. Titel der Arbeit  (In der Sprache, in der die Arbeit verfasst wurde)
    \vspace{2.0cm}
    {\Huge \textbf{Ein Lorem-Rahmenwerk}}\\ % bei langen Titeln ggf. Schriftgr��e auf \huge herunter setzen
    \vspace*{3mm}
    {\Huge \textbf{f�r Ipsum-Systeme}}\\
    \vspace*{3mm}
    {\Huge \textbf{-- ein Dolor-Ansatz}}\\
    \vspace{1.5cm}
% Vorname und Nachname des Bearbeiters/ der Bearbeiterin
    Vorname Nachname

    \vspace{5cm} % ggf. je nach Zeilenzahl und Schriftgr��e des Titels anpassen
    \includegraphics[width=2.4cm]{tum-info-logo}
\end{center}

\newpage

%%%%%%%%%%%%%%%%%%%%%%%%%%%%%%%
% R�ckseite Deckblatt

\thispagestyle{empty}
\cleardoublepage

%%%%%%%%%%%%%%%%%%%%%%%%%%%%%%%
% Erste Seite (Titelblatt)

\thispagestyle{empty}

\begin{center}

    \includegraphics[width=3cm]{tum-logo}\\
    \vspace{.5cm}
    {\Large \sc Technische Universit�t M�nchen}\\


    \vspace{.5cm}

    {\huge \sc Fakult�t f�r Informatik\\[1mm]}


    \vspace{1cm}

    {\Large \textbf{Diplomarbeit in Informatik}}\\ % oder SEP etc.

% Thema bzw. Titel der Arbeit  (In der Sprache, in der die Arbeit verfasst wurde)
    \vspace{1.5cm}
    {\huge \textbf{Ein Lorem-Rahmenwerk}}\\ % bei langen Titeln ggf. Schriftgr��e herunter setzen
    \vspace*{3mm}
    {\huge \textbf{f�r Ipsum-Systeme}}\\
    \vspace*{3mm}
    {\huge \textbf{-- ein Dolor-Ansatz}}\\

% die englische bzw. deutsche Entsprechung des Titels
    \vspace{1cm}
    {\huge \textbf{A Lorem Framework}}\\ % bei langen Titeln ggf. Schriftgr��e herunter setzen
    \vspace*{3mm}
    {\huge \textbf{for Ipsum Systems}}\\
    \vspace*{3mm}
    {\huge \textbf{-- a Dolor Approach}}\\
    \vspace{1cm}

    \parbox{1cm}{
      \begin{large}
        \begin{tabbing}
          Bearbeiter: \hspace{1.5cm}
            \=Vorname Nachname\\[2mm]
    Aufgabensteller: \>Prof. Dr. Dieter Kranzlm�ller\\[2mm]
    Betreuer: \>MNM-Team-Betreuer 1\\ % alphabetische Reihenfolge (Nachname)
    \>MNM-Team-Betreuer 2\\
    \>Externer Betreuer 1 (Firma)\\[5mm]
    Abgabedatum: \> 7. Juli 2077\\
        \end{tabbing}
      \end{large}
    }\\

    \vspace{.3cm}

    \includegraphics[width=2.4cm]{tum-info-logo}

\end{center}
 % Titelbl�tter TUM - auskommentiert lassen falls LMU-Arbeit
    \thispagestyle{empty}
    \cleardoublepage
    %
% LaTeX-Rahmen f�r Arbeiten am Lehrstuhl Hegering
%
% Harald Roelle, 2001, 2002
%
% basierend auf Arbeiten von Helmut Reiser, Boris Gruschke und Stephen Heilbronner
%

\newpage

\thispagestyle{empty}

\begin{large}

\vspace*{2cm}

\noindent
Hiermit versichere ich, dass ich die vorliegende Masterarbeit
selbst�ndig verfasst und keine anderen als die angegebenen Quellen
und Hilfsmittel verwendet habe.

\vspace{2cm}

\noindent
M�nchen, den 7. Juli 2077

\vspace{3cm}

\hspace*{7cm}%
\dotfill\\
\hspace*{8.5cm}%
\textit{(Unterschrift des Kandidaten)}

\end{large}
 % Erkl�rung (Arbeit selbstst�ndig verfasst) - auskommentieren falls TUM-Arbeit
%    \begin{large}

\vspace*{2cm}
\noindent
Ich versichere, dass ich diese Masterarbeit % (bzw. Master's Thesis)
selbst�ndig verfasst und nur die angegebenen Quellen und Hilfsmittel verwendet habe.

\vspace{2cm}

\noindent
M�nchen, den 7. Juli 2077

\vspace{3cm}

\hspace*{7cm}%
\dotfill\\
\hspace*{8.5cm}%
\textit{(Unterschrift des Kandidaten)}

\end{large}
 % Erkl�rung (Arbeit selbstst�ndig verfasst) - auskommentiert lassen falls LMU-Arbeit
    \thispagestyle{empty}
    \cleardoublepage
    \vspace*{2cm}

\begin{center}
    \textbf{Abstract}
\end{center}

\vspace*{1cm} 

\noindent The need for new programming abstractions in the advancements to the Exascale era has been widely recognized. In this, the cost of accessing data in Exascale systems is expected to be the dominant factor in terms of execution time and energy consumption.\\

\noindent With memory intensive workloads and unpredictable access patterns, graph algorithms are notoriously hard to implement and optimize for high performance distributed-memory systems.\\

\noindent The Partitioned Global Address Space (PGAS) programming model is considered as a promising approach to facilitate the shift from a compute-centric to a more data-centric focus and give application developers fine-grained control over data locality.\\

\noindent Several PGAS implementations of large-scale graph data structures are known but as of this writing, no generic programming abstraction for distributed graph data structures in PGAS exists.\\

\noindent This work examines PGAS graph containers as a representative use case for dynamic PGAS data structures and analyses suitable allocation and communication schemes with respect to optimized data locality.\\

\noindent Additionally, this work provides a reference implementation of the mentioned graph containers that is part of DASH, an implementation of PGAS as C++ template library for distributed containers and algorithms. The reference implementation is evaluated against existing PGAS approaches regarding scalability and programmability. % Abstract
    \thispagestyle{empty}
    \tableofcontents % Inhaltsverzeichnis

% ---------------------------------------------------------------
\mainmatter % die eigentliche Arbeit

\chapter{Introduction}
\section{Motivation}
\section{Scope}

\chapter{Background}
\section{Graph definition}
\section{Standard Template Library}
\section{Partitioned Global Address Space}
\section{DASH Library}

\chapter{Related Work}
\section{Shared Memory}
\subsection{STINGER}
\subsection{Ligra}
\section{Distributed Memory}
\subsection{Parallel Boost Graph Library}
\subsection{STAPL Parallel Graph Library}

\chapter{Container Concept}
\section{Interface semantics}
\section{Computational constraints/assumptions}
\section{Memory Space}
\section{Index Space}
\section{Iteration Space}

\chapter{Reference Implementation}

\chapter{Case studies}
\section{Static structure}
\subsection{Graph traversal}
\subsection{Shortest path evaluation}
\section{Dynamic Structure}
\subsection{Graph partitioning}
\subsection{De Bruijn Graph construction}

\chapter{Evaluation}
\section{Micro-benchmarks}

\chapter{Conclusion}
\section{Summary}
\section{Assessment}
\section{Outlook}

% ---------------------------------------------------------------
\backmatter % ab hier keine Nummerierung mehr
    \listoffigures
    \bibliographystyle{alphadin}
    \bibliography{./Bib/effe17}

\end{document}
